\documentclass{beamer}

\usepackage[french]{babel}
\usepackage[utf8]{inputenc}
\usepackage[T1]{fontenc}
\usepackage{amsmath, amsthm, amsfonts}
\usepackage{siunitx}
\usepackage{tikz}
\usepackage{hyperref}
%\usepackage[backend=biber, autocite=footnote]{biblatex}
\usepackage{xcolor}
\usepackage{caption}
\usepackage{booktabs}
\usepackage{mathtools}

\tikzset{>=latex}
\usetikzlibrary{calc,decorations.pathreplacing}
\sisetup{locale=FR, per-mode=symbol}

\newcommand{\abs}[1]{\left| #1 \right|}
%\renewcommand{\vec}[1]{\ensuremath{\overrightarrow{\boldsymbol{\mathrm{ #1 }}}}}
\newcommand{\rhat}{\vec{\hat{r}}}
\newcommand{\xhat}{\vec{i}}
\newcommand{\yhat}{\vec{j}}
\newcommand{\zhat}{\vec{k}}
\newcommand{\real}{\mathbb{R}}
\newcommand{\der}[2]{\frac{\mathrm{d}#1}{\mathrm{d}#2}}
\newcommand{\pder}[2]{\frac{\partial #1}{\partial #2}}
\newcommand{\dif}{\mathrm{d}}
\newcommand{\ddif}{\,\mathrm{d}}
\newcommand{\grad}{\vec{\nabla}}
\newcommand{\exemple}[1]{\begin{fullwidth}#1\end{fullwidth}}
\newcommand{\norm}[1]{\lVert #1 \rVert}
\newcommand{\vu}{\vec{u}}
\newcommand{\vv}{\vec{v}}
\newcommand{\vr}{\vec{r}}
\newcommand{\va}{\vec{a}}
\newcommand{\vF}{\vec{F}}
\newcommand{\vecxyz}[3]{#1 \xhat + #2 \yhat + #3 \zhat}
\newcommand{\vecxy}[2]{#1 \xhat + #2 \yhat}

\theoremstyle{definition}
\newtheorem*{defn}{Definition}



\newcommand{\teddybear}{
  % Right arm
  \fill[color=brown, shift={(1, 0.4)}, rotate=-60] (0, 0) ellipse (0.2 and 0.5);
  % Left arm
  \fill[color=brown, shift={(-1, 0.4)}, rotate=60] (0, 0) ellipse (0.2 and 0.5);
  % Right leg
  \fill[color=brown, shift={(0.6, -1.2)}, rotate=30] (0, 0) ellipse (0.2 and 0.5);
  % Left leg
  \fill[color=brown, shift={(-0.6, -1.2)}, rotate=-30] (0, 0) ellipse (0.2 and 0.5);
  % Body
  \fill[color=brown] (0, 0) ellipse (0.8 and 1);
  % Belly
  \fill[color=brown!50] (0, -0.2) ellipse (0.4 and 0.5);
  % Left outer ear
  \fill[color=brown, draw=brown!80!black] (-0.5, 1.9) circle (0.3);
  % Left inner ear
  \fill[color=brown!50] (-0.42, 1.8) circle (0.15);
  % Right outer ear
  \fill[color=brown, draw=brown!80!black] (0.5, 1.9) circle (0.3);
  % Right inner ear
  \fill[color=brown!50] (0.42, 1.8) circle (0.15);
  % Head
  \fill[color=brown, draw=brown!80!black] (0, 1.3) circle (0.6);
  % Eyes
  \fill[color=white] (-0.2, 1.4) circle (0.2);
  \fill[color=white] (0.2, 1.4) circle (0.2);
  \fill[color=black] (-0.2, 1.4) circle (0.1);
  \fill[color=black] (0.2, 1.4) circle (0.1);
}

\title{Questions de révision pour l'examen 1}


\begin{document}

\maketitle

\begin{frame}
  \frametitle{Analyse dimensionnelle}

  Lesquelles des équations suivantes sont homogènes.

  $[F] = ML/T^2$, $[a] = L/T^2$, $[m] = M$, $[v] = L/T$, $[t] = T$, $[x] = L$

  \begin{enumerate}
    \item \alert<2>{$\displaystyle{\frac{F}{m} = \frac{v}{t} }$}
    \item $\displaystyle{v - at = 4v + \frac{a}{x} t^2  }$
    \item $\displaystyle{x = vt + av}$
    \item \alert<2>{$\displaystyle{Fx = mv^2}$}
    \item $\displaystyle{tata = maFt}$
  \end{enumerate}
\end{frame}


\begin{frame}
  \frametitle{Vecteurs et scalaires}

  Quelles quantités parmis les suivantes sont des vecteurs?

  \begin{enumerate}
    \item \alert<2>{la vitesse}
    \item la distance parcourue
    \item la composante $y$ de l'accélération
    \item la norme du déplacement
    \item \alert<2>{l'accélération centripète}
    \item l'orientation de la position
  \end{enumerate}

\end{frame}


\begin{frame}
  \frametitle{Positif ou négatif?}

  Pour chacune des quantités suivantes déterminer si elle peut être négative,
  si elle est toujours plus grande ou égale à zéro, ou si on ne peut pas
  déterminer si elle est positive ou négative.

  \begin{enumerate}
    \item la vitesse \only<2>{\alert{vecteur, ni $+$ ni $-$}}
    \item la distance parcourue \only<2>{\alert{$\geq 0$}}
    \item la composante $y$ de l'accélération \only<2>{\alert{$<$, $=$ ou $>$}}
    \item la norme du déplacement \only<2>{\alert{$\geq 0$}}
    \item l'accélération centripète \only<2>{\alert{vecteur, ni + ni -}}
    \item la composante $x$ de la vitesse \only<2>{\alert{ $<$, $=$ ou $>$}}
  \end{enumerate}
\end{frame}


\begin{frame}
  \frametitle{Déplacement et distance parcourue}

  Un hibou vole d'un endroit à un autre. La distance parcourue par le hibou est
  \begin{enumerate}
    \item \alert<2>{plus grande ou égale au}
    \item toujours plus grande que le
    \item égale au
    \item plus petite ou égale au
    \item toujours plus petite que le
    \item soit plus petite ou plus grande que le
  \end{enumerate}
  module du déplacement.

\end{frame}


\begin{frame}
  \frametitle{Graphique position-temps I}

  La figure ci-dessous montre la position d'un ourson à cinq moments
  différents.
  \begin{center}
    \begin{tikzpicture}
      \draw[->] (0, 0) -- (9, 0) node[below] {$x$}; 
      \begin{scope}[shift={(0.5,   0.5)}, scale=0.2]\teddybear\end{scope}
      \begin{scope}[shift={(1.5, 0.5)},   scale=0.2]\teddybear\end{scope}
      \begin{scope}[shift={(3,   0.5)},   scale=0.2]\teddybear\end{scope}
      \begin{scope}[shift={(5,   0.5)},   scale=0.2]\teddybear\end{scope}
      \begin{scope}[shift={(8,   0.5)},   scale=0.2]\teddybear\end{scope}
      \node at (0.3, 1.3) {$t = \SI{4}{s}$};
      \node at (1.6, 1.3) {$t = \SI{3}{s}$};
      \node at (3  , 1.3) {$t = \SI{2}{s}$};
      \node at (5  , 1.3) {$t = \SI{1}{s}$};
      \node at (8  , 1.3) {$t = \SI{0}{s}$};
    \end{tikzpicture}
  \end{center}
  Lequel des graphiques suivants est un graphique de la position en fonction du
  temps plausible pour le mouvement de l'ourson.

  \begin{tikzpicture}[scale=0.6]
    \begin{scope}[shift={(0, 0)}]
      \draw[->] (0, 0) -- (3, 0) node[below] {$t$};
      \draw[->] (0, 0) -- (0, 3) node[left] {$x$};
      \draw (0.2, 2.5) parabola (2.5, 0.2);
      \node at (-0.5, 0.5) {A};
    \end{scope}
    \begin{scope}[shift={(4.5, 0)}]
      \draw[->] (0, 0) -- (3, 0) node[below] {$t$};
      \draw[->] (0, 0) -- (0, 3) node[left] {$x$};
      \draw (2.5, 2.5) parabola (0.2, 0.2);
      \node at (-0.5, 0.5) {B};
    \end{scope}
    \begin{scope}[shift={(9, 0)}]
      \draw[->] (0, 0) -- (3, 0) node[below] {$t$};
      \draw[->] (0, 0) -- (0, 3) node[left] {$x$};
      \draw (0.2, 0.2) parabola (2.5, 2.5);
      \node at (-0.5, 0.5) {C};
    \end{scope}
    \alert<2>{
    \begin{scope}[shift={(13.5, 0)}]
      \draw[->] (0, 0) -- (3, 0) node[below] {$t$};
      \draw[->] (0, 0) -- (0, 3) node[left] {$x$};
      \draw (2.5, 0.2) parabola (0.2, 2.5);
      \node at (-0.5, 0.5) {D};
    \end{scope}
    }
  \end{tikzpicture}

\end{frame}

\begin{frame}
  \frametitle{Graphique position-temps II}

  Le graphique suivant montre la position en fonction du temps de deux trains
  qui voyagent sur des voies parallèles.  Pour les temps plus grands que $t =
  \SI{0}{s}$, lequel des énoncés suivants est vrai.

  \begin{center}
  \begin{tikzpicture}[scale=0.7]
    \draw[->] (0, 0) -- (3, 0) node[below] {$t$};
    \draw[->] (0, 0) -- (0, 3) node[left] {$x$};
    \draw (0, 0) to[bend left] (2.9, 2.6) node[right] {$B$};
    \draw (0, 0) -- (2.9, 2.9) node[anchor=south west] {$A$};
    \draw[dashed] (2.5, 2.5) -- (2.5, 0) node[below] {$t_B$};
    \alert<2>{
      \draw[shift={(0.7, 1.4)}] (-0.5, -0.5) -- (1, 1);
      \draw[dashed] (1, 0) -- (1, 1.6);
    }
  \end{tikzpicture}
  \end{center}

  \begin{enumerate}
    \item Au temps $t_B$, le module de la vitesse des deux trains est le même.
    \item Les deux trains vont de plus en plus vite.
    \item \alert<2>{Le module de la vitesse des deux trains est le même à un
        instant entre \SI{0}{s} et $t_B$.}
    \item Quelque part sur le graphique, les deux trains ont la même
      accélération.
  \end{enumerate}

\end{frame}

\begin{frame}
  \frametitle{Chute libre I}

  On lance une balle verticalement vers le haut.  Au point le plus haut de sa
  trajectoire, lequel des énoncés suivants est vrai.

  \begin{enumerate}
    \item les modules de la vitesse et de l'accélération sont nuls
    \item le module de la vitesse est différent de zéro et celui de
      l'accélération est nul
    \item \alert<2>{le module de la vitesse est nul et celui de l'accélération
        est différent de zéro}
    \item les modules de la vitesse et de l'accélération sont tous les deux
      différents de zéro
  \end{enumerate}
\end{frame}


\begin{frame}
  \frametitle{Graphique vitesse-temps}

  Le graphique ci-dessous illustre la composante $x$ de la vitesse en fonction
  du temps pour un hibou qui vole en ligne droite.

  \begin{center}
  \begin{tikzpicture}[scale=1]
    \draw[->] (0, 0) -- (3, 0) node[below] {$t$};
    \draw[->] (0, 0) -- (0, 1.5) node[left] {$v_x$};
    \draw (0.5, 0) -- (1, 1) -- (1.5, 0) -- (2, 1) -- (2.5, 0);
    \draw[dashed] (0, 1) node[left] {$v_{max}$} -- (2, 1);
    \node[below] at (0.5, 0) {$t_0$};
    \node[below] at (1.5, 0) {$t_1$};
    \node[below] at (2.5, 0) {$t_2$};
  \end{tikzpicture}
  \end{center}

  Laquelle des expressions suivantes donne le déplacement entre $t_0$ et $t_2$?

  \begin{enumerate}
    \item $\displaystyle{\Delta x = (t_1 - t_0)v_{max} + (t_2 - t_0)v_{max} }$
    \item $\displaystyle{\Delta x = (t_2 - t_0)v_{max} }$
    \item \alert<2>{$\displaystyle{\Delta x = \frac{(t_2 - t_0)v_{max}}{2} }$}
    \item $\displaystyle{\Delta x = \frac{1}{2} (t_0 + t_1 + t_2) v_{max} }$
  \end{enumerate}


\end{frame}


\begin{frame}
  \frametitle{Projectile}
  
  Un joueur de baseball frappe une balle et lui donne une vitesse initiale de
  module \SI{50}{\meter\per\second} faisant un angle de \SI{30}{\degree} avec
  l'horizontale.  Combien de temp est nécessaire pour que la balle atteigne
  son point le plus haut? (Supposer que $g = \SI{10}{\meter\per\second\squared}$.)

  \begin{enumerate}
    \item \SI{2.0}{s}
    \item \alert<2>{\SI{2.5}{s}}
    \item \SI{3.0}{s}
    \item \SI{4.0}{s}
    \item \SI{5.0}{s}
    \item \SI{6.0}{s}
  \end{enumerate}

\end{frame}


\end{document}


