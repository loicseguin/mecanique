\chapter{Rotation des corps solides}


\section{Cinématique de rotation}

Très souvent, l'approximation de la particule n'est pas appropriée pour décrire
une situation physique.  On n'a qu'à penser au fonctionnement d'une roue de
bicyclette, à un levier, à la rotation d'une planète autour de son axe ou au
mouvement d'un bâton de hockey.  Dans ces situations, on s'intéresse à la façon
dont diverses parties d'un même objet se déplacent.  Non seulement chaque
partie de l'objet se déplace à une vitesse différente, mais l'endroit où les
forces agissent sur l'objet joue un rôle important.  Pour décrire complètement
le mouvement, on décrit le mouvement de translation du centre de masse et le
mouvement de rotation par rapport l'axe autout duquel l'objet tourne.


