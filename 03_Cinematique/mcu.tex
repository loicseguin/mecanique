\documentclass{tufte-handout}

\usepackage[french]{babel}
\usepackage[utf8]{inputenc}
\usepackage[T1]{fontenc}
\usepackage{amsmath, amsthm, amsfonts}
\usepackage{siunitx}
\usepackage{tikz}
\usepackage{hyperref}
%\usepackage[backend=biber, autocite=footnote]{biblatex}
\usepackage{xcolor}
\usepackage{caption}
\usepackage{booktabs}
\usepackage{mathtools}

\tikzset{>=latex}
\usetikzlibrary{calc,decorations.pathreplacing}
\sisetup{locale=FR, per-mode=symbol}

\newcommand{\abs}[1]{\left| #1 \right|}
%\renewcommand{\vec}[1]{\ensuremath{\overrightarrow{\boldsymbol{\mathrm{ #1 }}}}}
\newcommand{\rhat}{\vec{\hat{r}}}
\newcommand{\xhat}{\vec{i}}
\newcommand{\yhat}{\vec{j}}
\newcommand{\zhat}{\vec{k}}
\newcommand{\real}{\mathbb{R}}
\newcommand{\der}[2]{\frac{\mathrm{d}#1}{\mathrm{d}#2}}
\newcommand{\pder}[2]{\frac{\partial #1}{\partial #2}}
\newcommand{\dif}{\mathrm{d}}
\newcommand{\ddif}{\,\mathrm{d}}
\newcommand{\grad}{\vec{\nabla}}
\newcommand{\exemple}[1]{\begin{fullwidth}#1\end{fullwidth}}
\newcommand{\norm}[1]{\lVert #1 \rVert}
\newcommand{\vu}{\vec{u}}
\newcommand{\vv}{\vec{v}}

\theoremstyle{definition}
\newtheorem*{defn}{Definition}



\title{Cinématique du mouvement circulaire uniforme}
\date{}

\begin{document}

\maketitle
\vspace{0.5cm}

\section{Mouvement circulaire uniforme}

Un \textbf{mouvement circulaire uniforme} est un mouvement en deux dimensions
dont la trajectoire est un cercle et au cours duquel le module de la vitesse
est constant.  Pour ce type de mouvement, il existe une relation simple (bien
que sa démonstration ne le soit pas) entre le module de l'accélération, le
module de la vitesse et le rayon de la trajectoire.

\begin{marginfigure}
  \begin{tikzpicture}[scale=1]
    \draw (0, 0) circle (2);
    \draw[very thick, ->] (0, 0) -- node[anchor=north west] {$\vec{r}_1$} (30:2);
    \draw[very thick, ->] (0, 0) -- node[anchor=south east] {$\vec{r}_2$} (60:2);
    \draw[very thick, ->] (30:2) -- node[right] {$\vec{v}_1$} ++(120:2);
    \draw[very thick, ->] (60:2) -- node[above] {$\vec{v}_2$} ++(150:2);
    \fill (30:2) circle (2pt);
    \fill (60:2) circle (2pt);
  \end{tikzpicture}
\end{marginfigure}
La figure ci-contre montre la position et la vitesse d'une particule en
mouvement circulaire uniforme à deux moments distincts séparés par un
intervalle de temps $\Delta t$.  L'origine du système de coordonnées est placée
au centre du cercle.  La trajectoire a un rayon $R$ et le module de la vitesse
est $v$.  Par conséquent,
\[
  r_1 = r_2 = R
\]
et
\[
  v_1 = v_2 = v.
\]
L'objectif est de déterminer l'\textbf{accélération instantannée} pour ce
mouvement.  La première étape est de déterminer l'orientation de
l'accélération, et la seconde, de déterminer son module.


\subsection{Orientation de l'accélération d'un MCU}

Le diagramme ci-contre représente les vitesses aux mêmes deux moments que dans
la figure précédente.  Par contre, l'origine de ces deux vitesses est placée au
point $A$ qui correspond à la position $\vec{r}_1$.
\begin{marginfigure}
  \begin{tikzpicture}[scale=1.2]
    \clip (-0.5, -0.5) rectangle (5, 5);
    \draw (0, 0) circle (3);
    \draw (0, 0) -- (60:3);
    \draw[very thick, ->] (30:3) -- ++(150:2) node[anchor=east] {$\vec{v}_2$};
    \draw (0, 0) -- (30:3);
    \draw[very thick, ->] (30:3) -- ++(120:2) node[above] {$\vec{v}_1$};
    \draw[very thick, ->] ($(30:3) + (120:2)$) --
                          node[anchor=south east] {$\Delta\vv$}
                          ($(30:3) + (150:2)$);
    \draw[dashed] (0, 0) -- (45:3.2);
    \fill[shift={(60:2.45)}, rotate=-30] (0, 0) rectangle (0.15, 0.15);
    \fill[shift={(30:2.85)}, rotate=30] (0, 0) rectangle (0.15, 0.15);
    \draw[fill=green, opacity=0.5] (0, 0) -- (30:0.5) arc (30:45:0.5) -- cycle;
    \filldraw[fill=green, opacity=0.5] (0, 0) -- (45:0.7) arc (45:60:0.7) -- cycle;
    \filldraw[fill=red, opacity=0.5] (45:2.67) -- (45:2.45) arc (215:155:0.27) -- cycle;
    \filldraw[fill=red, opacity=0.5] (45:2.7) -- (45:2.85) arc (35:-15:0.23) -- cycle;
    \filldraw[fill=red, opacity=0.5] (45:3.1) -- (45:2.95) arc (-155:-55:0.12) -- cycle;
    \node[below] at (0, 0) {$O$};
    \node[right] at (30:3) {$A$};
    \node[] at (45:3.5) {$B$};
    \node[below] at (45:2.7) {$C$};
    \node[left] at (60:2.5) {$D$};
  \end{tikzpicture}
\end{marginfigure}
La droite en pointillées est la bissectrice de l'angle formé par les deux
vecteurs position $\vec{r}_1$ et $\vec{r}_2$, et, par conséquent, les deux
angles verts sont égaux.  Le triangle $ODC$ est rectangle parce que $\vv_2$ est
perpendiculaire à $\vec{r}_2$.  L'angle rouge dans ce triangle est donc
\[
  \tikz{\filldraw[fill=red, opacity=0.5]
          (0, 0) -- (0.4, 0) arc (0:60:0.4) -- cycle;} = 
  \SI{90}{\degree} -
  \tikz{\filldraw[fill=green, opacity=0.5]
          (0, 0) -- (0.4, 0) arc (0:30:0.4) -- cycle;}.
\]
Puisque l'angle $\angle ACB$ est opposé par le sommet à l'angle $\angle OCD$,
ces deux angles sont égaux.

Le triangle $OAB$ est rectangle parce que $\vv_1$ est
perpendiculaire à $\vec{r}_1$.  L'angle rouge dans ce triangle est donc aussi
\[
  \tikz{\filldraw[fill=red, opacity=0.5]
          (0, 0) -- (0.4, 0) arc (0:60:0.4) -- cycle;} = 
  \SI{90}{\degree} -
  \tikz{\filldraw[fill=green, opacity=0.5]
          (0, 0) -- (0.4, 0) arc (0:30:0.4) -- cycle;}.
\]
Tous les angles rouges dans la figure sont donc égaux.

Le triangle $ABC$ est par conséquent isocèle.  De plus, il partage un angle
avec le triangle formé par $\vv_1$, $\vv_2$ et $\Delta \vv$, qui est aussi
isocèle.  Par conséquent, $\Delta \vv$ est parallèle avec la droite bissectrice
$OB$.  Si l'angle $2\tikz{\filldraw[fill=green, opacity=0.5] (0, 0) -- (0.4, 0)
  arc (0:30:0.4) -- cycle;}$ est petit (lire, tend vers zéro), l'accélération
instantannée à mi-chemin entre les positions $\vr_1$ et $\vr_2$ est
approximativement égale à
\[
  \vec{a} \approx \frac{\Delta \vv}{\Delta t} 
\]
L'accélération instantannée est donc parallèle avec la droite bissectrice $OB$.
Autrement dit, elle pointe vers le centre du cercle.

En résumé, \textbf{l'accélération dans un MCU pointe toujours vers le centre du
cercle de la trajectoire}.  On l'appelle donc une \textbf{accélération
centripète}.



\subsection{Module de l'accélération d'un MCU}

Maintenant que nous connaissons l'orientation du vecteur accélération, il ne reste
qu'à trouver son module.  Pour cela, nous utiliserons la définition de
l'accélération moyenne puis nous obtiendrons l'accélération instantannée en
passant à la limite lorsque l'intervalle de temps $\Delta t$ tend vers zéro.

La figure ci-contre montre encore une fois les vecteurs vitesse aux deux
moments considérés.  En prolongeant le segment correspondant à $\vr_1$ et celui
correspondant à $\vv_2$, on forme deux triangles indiqués en rouge et en bleu
sur la figure.  Ces deux triangles sont rectangles puisque les vecteurs vitesse
sont tangent au cercle et donc perpendiculaires aux vecteurs position.
\begin{marginfigure}
  \begin{tikzpicture}[scale=1.3]
    \clip (-0.5, -0.5) rectangle (5, 5);
    \draw (0, 0) circle (3);
    \draw (0, 0) -- (60:3);
    \draw[very thick, ->] (60:3) -- ++(150:2) node[above] {$\vec{v}_2$};
    \draw[dashed] (60:3) -- ++(-30:1.76);
    \draw (0, 0) -- (30:3);
    \draw[very thick, ->] (30:3) -- ++(120:2) node[above] {$\vec{v}_1$};
    \draw[dashed] (30:3) -- (30:3.46);
    \draw[opacity=0.5, blue, line width=2pt] (0, 0) -- (60:3) -- (30:3.46) -- cycle;
    \begin{scope}
      \clip (30:3.46) -- (30:3) -- ++(120:0.8) -- cycle;
      \draw[opacity=0.5, red, line width=4pt] (30:3.46) -- (30:3) -- ++(120:0.8) -- cycle;
    \end{scope}
    \filldraw[fill=green, opacity=0.5] (0, 0) -- (30:0.4) arc (30:60:0.4) -- cycle;
    \filldraw[fill=green, opacity=0.5, shift={(-45:0.2)}] (45:3.1) -- (40:3.05) arc (-50:-30:0.4) -- cycle;
  \end{tikzpicture}
\end{marginfigure}
Les deux triangles rectangles partagent un angle autre que l'angle de
\SI{90}{\degree}, donc ils sont semblables et tous leurs angles sont égaux.  Si
on appelle $\Delta \theta$ l'angle \tikz{\filldraw[fill=green, opacity=0.5] (0,
  0) -- (30:0.4) arc (30:60:0.4) -- cycle;} entre $\vr_1$ et $\vr_2$, alors
l'angle \tikz{\filldraw[fill=green, opacity=0.5] (0, 0) -- (30:0.4) arc
  (30:60:0.4) -- cycle;} dans le triangle rouge est aussi égal à $\Delta
\theta$.  L'angle entre les vecteurs $\vv_1$ et $\vv_2$ est opposé par le
sommet à l'angle \tikz{\filldraw[fill=green, opacity=0.5] (0, 0) -- (30:0.4) arc
  (30:60:0.4) -- cycle;} dans le triangle rouge et il est donc aussi égal à
$\Delta \theta$.

Par conséquent, l'angle entre les vecteurs $\vr_1$ et $\vr_2$ est le même que
l'angle entre les vecteurs $\vv_1$ et $\vv_2$.

Les vecteurs $\vv_1$ et $\vv_2$ ont le même module.  S'ils ont la même origine,
ils forment donc deux rayons d'un cercle de rayon $v$, comme sur la figure
ci-contre.  La longueur du vecteur $\Delta \vv$ est approximativement égale à
la longueur de l'arc de cercle indiqué en rouge.  Cette approximation devient
une égalité stricte si on prend la limite de $\Delta \theta$ qui tend vers
zéro.
\begin{marginfigure}
  \begin{tikzpicture}[scale=1.3]
    \clip (0.5, -0.5) rectangle (-3.5, 3.5);
    \draw (0, 0) circle (3);
    \draw[very thick, red] (120:3) arc (120:150:3);
    \draw[very thick, ->] (0, 0) -- node[anchor=north east] {$\vec{v}_2$} (150:3);
    \draw[very thick, ->] (0, 0) -- node[anchor=south west] {$\vec{v}_1$} (120:3);
    \draw (120:0.8) arc (120:150:0.8);
    \node at (135:1.2) {$\Delta \theta$};
    \draw[very thick, ->] (120:3) -- node[anchor=north west] {$\Delta\vec{v}$} (150:3);
    \node[red] at (135:3.15) {$s$};
\end{tikzpicture}
\end{marginfigure}
La longueur d'un arc de cercle sous-tendu par un angle $\Delta \theta$ est
égale au rayon du cercle multiplié par l'angle en radian:
\[
  {\color{red}s} = v\Delta \theta.
\]
Par conséquent, le module du vecteur changement de vitesse est
approximativement
\[
  \Delta v \approx v\Delta\theta.
\]
Le module de l'accélération instantannée est approximativement égal au module
de l'accélération moyenne donc
\begin{align}
  a &\approx \frac{\Delta v}{\Delta t} \nonumber\\
  a &\approx \frac{v \Delta\theta}{\Delta t} \label{eqn:mcu_a} 
\end{align}
Il ne reste qu'à éliminer le $\Delta \theta / \Delta t$ de cette expression.
Dans le cercle de la trajectoire, les vecteurs positions $\vr_1$ et $\vr_2$
forment deux rayons.  Le déplacement $\Delta \vr$ a un module approximativement
égal à la longueur de l'arc de cercle $s_r$
\[
  \Delta r \approx R \Delta \theta.
\]
\begin{marginfigure}
  \begin{tikzpicture}[>=latex]
    \clip (-0.5, -0.5) rectangle (3.5, 3.5);
    \draw (0, 0) circle (3);
    \draw[very thick, red] (30:3) arc (30:60:3);
    \draw[very thick, ->] (0, 0) -- node[anchor=south east] {$\vec{r}_2$} (60:3);
    \draw[very thick, ->] (0, 0) -- node[anchor=north west] {$\vec{r}_1$} (30:3);
    \draw (30:0.8) arc (30:60:0.8);
    \node at (45:1.2) {$\Delta \theta$};
    \draw[very thick, ->] (30:3) -- node[anchor=north east] {$\Delta\vec{r}$} (60:3);
    \node[red] at (45:3.25) {$s_r$};
  \end{tikzpicture}
\end{marginfigure}
Par conséquent, le module de la vitesse instantannée, $v$, est
approximativement égal à
\[
  v \approx \frac{R\Delta \theta}{\Delta t}.
\]
Autrement dit,
\[
  \frac{\Delta \theta}{\Delta t} = \frac{v}{R}.
\]
En remplaçant cette expression dans l'équation \ref{eqn:mcu_a}, nous obtenons
le module de l'accélération
\[
  a = \frac{v^2}{R} 
\]

Il est important de noter que ce résultat est \textbf{exact}, ce n'est pas une
appoximation.  La démonstration que nous en avons faite utilise plusieurs
approximations, mais il est possible de démontrer que ces approximations
deviennent des résultats exacts lorsque nous prenons la limite de $\Delta t$
qui tend vers zéro.


\subsection{Résumé MCU}

Dans un mouvement circulaire uniforme, la vitesse a un module constant $v$ et
la trajectoire est un cercle de rayon $R$.  L'accélération est centripète,
c'est-à-dire qu'elle pointe toujours vers le centre du cercle.  Enfin, le
module de l'accélération a la valeur constante $a = v^2/R$.

\end{document}

