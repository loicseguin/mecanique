\documentclass{beamer}

\usepackage[french]{babel}
\usepackage[utf8]{inputenc}
\usepackage[T1]{fontenc}
\usepackage{amsmath, amsthm, amsfonts}
\usepackage{siunitx}
\usepackage{tikz}
\usepackage{hyperref}
%\usepackage[backend=biber, autocite=footnote]{biblatex}
\usepackage{xcolor}
\usepackage{caption}
\usepackage{booktabs}
\usepackage{mathtools}

\tikzset{>=latex}
\usetikzlibrary{calc,decorations.pathreplacing}
\sisetup{locale=FR, per-mode=symbol}

\newcommand{\abs}[1]{\left| #1 \right|}
%\renewcommand{\vec}[1]{\ensuremath{\overrightarrow{\boldsymbol{\mathrm{ #1 }}}}}
\newcommand{\rhat}{\vec{\hat{r}}}
\newcommand{\xhat}{\vec{i}}
\newcommand{\yhat}{\vec{j}}
\newcommand{\zhat}{\vec{k}}
\newcommand{\real}{\mathbb{R}}
\newcommand{\der}[2]{\frac{\mathrm{d}#1}{\mathrm{d}#2}}
\newcommand{\pder}[2]{\frac{\partial #1}{\partial #2}}
\newcommand{\dif}{\mathrm{d}}
\newcommand{\ddif}{\,\mathrm{d}}
\newcommand{\grad}{\vec{\nabla}}
\newcommand{\exemple}[1]{\begin{fullwidth}#1\end{fullwidth}}
\newcommand{\norm}[1]{\lVert #1 \rVert}
\newcommand{\vu}{\vec{u}}
\newcommand{\vv}{\vec{v}}
\newcommand{\vr}{\vec{r}}
\newcommand{\va}{\vec{a}}
\newcommand{\vF}{\vec{F}}
\newcommand{\vecxyz}[3]{#1 \xhat + #2 \yhat + #3 \zhat}
\newcommand{\vecxy}[2]{#1 \xhat + #2 \yhat}

\theoremstyle{definition}
\newtheorem*{defn}{Definition}



\begin{document}

\begin{frame}
  \frametitle{Voiture dans un virage I}

  Une voiture prend une bretelle de sortie sur l'autoroute à une vitesse de
  module \SI{80}{\kilo\meter\per\hour}.  Si le rayon de courbure de la bretelle
  est de \SI{50}{\meter}, calculer le module de l'accélération centripète sur
  la voiture.

\end{frame}


\begin{frame}
  \frametitle{Voiture dans un virage II}

On considère une voiture qui entre sur l'autoroute par une bretelle circulaire
de rayon égal à \SI{50}{\meter}.  Entre \SI{0}{\second} et \SI{3}{\second} la
voiture passe de \SI{80}{\kilo\meter\per\hour} à \SI{100}{\kilo\meter\per\hour}.
\begin{enumerate}
  \item Dessiner qualitativement le vecteur vitesse à \SI{0}{\second} et à
    \SI{3}{\second}.
  \item Dessiner qualitativement l'accélération centripète, l'accélération
    tangentielle et l'accélération.
  \item Déterminer le module de l'accélération.
\end{enumerate}

\uncover<2->{
\begin{center}
  \begin{tikzpicture}
    \fill[black!40] (0, 0) arc (180:0:2) -- (3.5, 0)
          arc (0:180:1.5) -- (0, 0);
    \draw[dashed, yellow] (0.25, 0) arc (180:0:1.75);
    \fill[shift={(0.25, 0.5)}, rotate=-20] (0, 0) rectangle ++(0.2, 0.4);
    \fill[shift={(3, 1.25)}, rotate=55] (0, 0) rectangle ++(0.2, 0.4);
    \fill (2, 0) circle (2pt);
    \uncover<3->{
      \draw[very thick, blue, ->] (0.43, 0.7) -- ++(70:1) node[anchor=south east] {$\vec{v}_0$};
      \draw[very thick, blue, ->] (3, 1.38) -- ++(-35:1.5) node[anchor=south west] {$\vec{v}_1$};
    }
    \uncover<4->{
      \draw[very thick, red, ->] (1.4, 1.64) -- ++(20:0.5) node[anchor=south] {$\vec{a}_t$};
      \draw[very thick, red, ->] (1.4, 1.64) -- ++(-70:1.5) node[anchor=east] {$\vec{a}_r$};
      \coordinate (aa) at ($(1.4, 1.64) + (20:0.5) + (-70:1.5)$);
      \draw[very thick, red, ->] (1.4, 1.64) -- (aa) node[anchor=north west] {$\vec{a}$};
      \draw[densely dashed] ($(1.4, 1.64) + (20:0.5)$) -- (aa);
      \draw[densely dashed] ($(1.4, 1.64) + (-70:1.5)$) -- (aa);
    }
  \end{tikzpicture}
\end{center}
}

  
\end{frame}

\end{document}


