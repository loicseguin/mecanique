\documentclass{beamer}

\usepackage[french]{babel}
\usepackage[utf8]{inputenc}
\usepackage[T1]{fontenc}
\usepackage{amsmath, amsthm, amsfonts}
\usepackage{siunitx}
\usepackage{tikz}
\usepackage{hyperref}
%\usepackage[backend=biber, autocite=footnote]{biblatex}
\usepackage{xcolor}
\usepackage{caption}
\usepackage{booktabs}
\usepackage{mathtools}

\tikzset{>=latex}
\usetikzlibrary{calc,decorations.pathreplacing}
\sisetup{locale=FR, per-mode=symbol}

\newcommand{\abs}[1]{\left| #1 \right|}
%\renewcommand{\vec}[1]{\ensuremath{\overrightarrow{\boldsymbol{\mathrm{ #1 }}}}}
\newcommand{\rhat}{\vec{\hat{r}}}
\newcommand{\xhat}{\vec{i}}
\newcommand{\yhat}{\vec{j}}
\newcommand{\zhat}{\vec{k}}
\newcommand{\real}{\mathbb{R}}
\newcommand{\der}[2]{\frac{\mathrm{d}#1}{\mathrm{d}#2}}
\newcommand{\pder}[2]{\frac{\partial #1}{\partial #2}}
\newcommand{\dif}{\mathrm{d}}
\newcommand{\ddif}{\,\mathrm{d}}
\newcommand{\grad}{\vec{\nabla}}
\newcommand{\exemple}[1]{\begin{fullwidth}#1\end{fullwidth}}
\newcommand{\norm}[1]{\lVert #1 \rVert}
\newcommand{\vu}{\vec{u}}
\newcommand{\vv}{\vec{v}}

\theoremstyle{definition}
\newtheorem*{defn}{Definition}



\begin{document}

\begin{frame}
  \frametitle{Addition de vecteurs I}
Schklaktonek lance une rondelle de hockey du point $A$ au point $B$ où Pianotta
la reçoit et l'envoie vers le filet, au point $C$.  Quel est le déplacement
total de la rondelle?

\vspace{0.5cm}

\begin{center}
\begin{tikzpicture}
  \begin{scope}[opacity=0.5]
    \draw[rounded corners=25] (0, 0) rectangle (6, 3);
    \draw[red] (0.5, 0.15) -- (0.5, 2.85);
    \draw[red] (5.5, 0.15) -- (5.5, 2.85);
    \draw[red, thick] (3, 0.01) -- (3, 2.99);
    \draw[blue, thick] (2.2, 0.01) -- (2.2, 2.99);
    \draw[blue, thick] (3.8, 0.01) -- (3.8, 2.99);
    \draw[blue] (3, 1.5) circle (0.5);
  \end{scope}
  \draw[fill=black] (1, 0.3) circle (1pt);
  \node[above] (A) at (1, 0.3) {$A$};
  \draw[fill=black] (3.2, 2.3) circle (1pt);
  \node[above] (B) at (3.2, 2.3) {$B$};
  \draw[fill=black] (5.5, 1.5) circle (1pt);
  \node[above] (C) at (5.5, 1.5) {$C$};
  \draw[very thick, ->] (1, 0.3) -- (3.2, 2.3);
  \draw[very thick, ->] (3.2, 2.3) -- (5.5, 1.5);
  \uncover<2>{
    \draw[very thick, ->, green!60!black] (1, 0.3) --(5.5, 1.5);
    \node[fill=white, text=black] at (2.0, 1.8) {$\vec{u}$};
    \node[fill=white, text=black] at (4.5, 2.3) {$\vec{v}$};
    \node[text=green!60!black, fill=white] at (3.5, 0.5) {$\vec{u} + \vec{v}$};
  }
\end{tikzpicture}
\end{center}

\end{frame}


\begin{frame}
  \frametitle{Addition de vecteurs II}
En supposant que $\norm{\vec{a}} = \norm{\vec{b}} = \norm{\vec{c}}$, faite
l'addition des vecteurs suivants, c'est-à-dire calculez $\vec{a} +
\vec{b} + \vec{c}$.

\vspace{0.5cm}

\only<1>{
  \begin{tikzpicture}
    \draw[->] (0, 0) -- node[left] {$\vec{a}$} (2, 4);
    \draw[dashed] (0, 0) -- (2, 0);
    \draw (1, 0) arc (0:63:1);
    \node at (1.3, 0.8) {\SI{60}{\degree}};
    \begin{scope}[shift={(2, 0)}]
      \draw[->] (2, 4) -- node[left] {$\vec{b}$} (4, 0);
      \draw[dashed] (2, 4) -- (4, 4);
      \draw (3, 4) arc (0:-63:1);
      \node at (3.3, 3.3) {\SI{60}{\degree}};
    \end{scope}
    \begin{scope}[shift={(7, 1)}]
      \draw[->] (4, 0) -- node[below] {$\vec{c}$} (0, 0);
    \end{scope}
  \end{tikzpicture}
}

\only<2>{
\begin{columns}
  \column{0.5\textwidth}
  \begin{tikzpicture}
    \draw[->] (0, 0) -- node[left] {$\vec{a}$} (2, 4);
    \draw (1, 0) arc (0:63:1);
    \node at (1.3, 0.8) {\SI{60}{\degree}};
      \draw[->] (2, 4) -- node[left] {$\vec{b}$} (4, 0);
      \draw[dashed] (2, 4) -- (4, 4);
      \draw (3, 4) arc (0:-63:1);
      \node at (3.3, 3.3) {\SI{60}{\degree}};
      \draw[->] (4, 0) -- node[below] {$\vec{c}$} (0, 0);
  \end{tikzpicture}

  \column{0.5\textwidth}
  $\vec{a} + \vec{b} + \vec{c} = \vec{0}$
\end{columns}
}

\end{frame}


\begin{frame}
  \frametitle{Addition algébrique de vecteurs I}

  \only<1>{
  Pour se rendre à l'endroit où il doit sauver des naufragés, un pilote
  d'hélicoptère décolle de l'aéroport et parcourt \SI{20}{\kilo\meter} vers le
  nord-ouest.  Il bifurque ensuite de \SI{60}{\degree} vers sa droite et
  parcourt un autre \SI{45}{\kilo\meter}.

  En utilisant la décomposition des vecteurs en fonction des vecteurs
  unitaires, déterminer le déplacement total effectué par le pilote.
  }

  \only<2>{
    {\tiny
      Pour se rendre à l'endroit où il doit sauver des naufragés, un pilote
      d'hélicoptère décolle de l'aéroport et parcourt \SI{20}{\kilo\meter} vers le
      nord-ouest.  Il bifurque ensuite de \SI{60}{\degree} vers sa droite et
      parcourt un autre \SI{45}{\kilo\meter}.

      En utilisant la décomposition des vecteurs en fonction des vecteurs
      unitaires, déterminer le déplacement total effectué par le pilote.
    }
  }

  \uncover<2>{
    \begin{columns}
      \column{0.3\textwidth}
      \begin{tikzpicture}[scale=0.7]
        \draw[->] (-2, 0) -- (1, 0) node[right] {$x$};
        \draw[->] (0, -1) -- (0, 6) node[left] {$y$};
        \coordinate (A) at (0, 0);
        \coordinate (B) at ($(0, 0) + (135:2)$);
        \coordinate (C) at ($(B) + (75:4.5)$);
        \draw[dashed] (A) -- ($1.4*(B)$);
        \draw[->, very thick] (A) -- node[below] {$\vu$} (B);
        \draw (0, 0.6) arc (90:135: 0.6);
        \node at (-0.3, 0.8) {\SI{45}{\degree}};
        \draw ($(B) + 0.1*(75:4.5)$) arc (75:135:0.4);
        \node at (-1.6, 2.2) {\SI{60}{\degree}};
        \draw[->, very thick] (B) -- node[left] {$\vv$} (C);
      \end{tikzpicture}

      \column{0.8\textwidth}
      \vspace{-0.4cm}
      \begin{eqnarray*}
        u_x &= -u \sin \SI{45}{\degree} &= -\SI{20}{\kilo\meter} \sin
          \SI{45}{\degree} = -\SI{14.14213}{\kilo\meter} \\
        u_y &=  u \cos \SI{45}{\degree} &=  \SI{20}{\kilo\meter} \cos
          \SI{45}{\degree} = \SI{14.14213}{\kilo\meter} \\
      \end{eqnarray*}
      \vspace{-1.2cm}
      \[
        \vu = (-\num{14.1}\, \xhat + \num{14.1}\, \yhat) \si{\kilo\meter}
      \]
      Pour $\vv$, on note que l'angle entre l'axe des $x$ positifs et
      le vecteur est \SI{75}{\degree}.
      \vspace{-0.4cm}
      \begin{eqnarray*}
        v_x &= -v \cos \SI{75}{\degree} &= -\SI{45}{\kilo\meter} \cos
          \SI{75}{\degree} = \SI{11.64685}{\kilo\meter} \\
        v_y &=  v \sin \SI{75}{\degree} &=  \SI{45}{\kilo\meter} \sin
          \SI{75}{\degree} = \SI{43.46666}{\kilo\meter} \\
      \end{eqnarray*}
      \vspace{-1.2cm}
      \[
        \vv = (\num{11.6}\, \xhat + \num{43.5}\, \yhat) \si{\kilo\meter}
      \]
      \vspace{-1.0cm}
      \begin{align*}
        \vu + \vv &= (-\num{14.14213}\, \xhat + \num{14.14213}\, \yhat) \si{\kilo\meter} \\
                  &{}  +(\num{11.64685}\, \xhat + \num{43.46666}\, \yhat)
                    \si{\kilo\meter} \\
                  &= (-\num{2.50}\, \xhat + \num{57.6}\, \yhat) \si{\kilo\meter}
      \end{align*}
    \end{columns}
  }
\end{frame}


\begin{frame}
  \frametitle{Addition algébrique de vecteurs II}

  En général, est-ce que $\norm{\vv} + \norm{\vu} = \norm{\vv + \vu}$?

  \vspace{1cm}

  \uncover<2>{
    À partir de la décomposition obtenue à la question sur le pilote
    d'hélicoptère pour chacun des
    deux déplacements, $\vu$ et $\vv$, calculer les quantités suivantes:

    \begin{enumerate}
      \item $\norm{\vv} + \norm{\vu}$
      \item $\norm{\vv + \vu}$
    \end{enumerate}
  }
\end{frame}


\begin{frame}
  \frametitle{Mutliplication par un scalaire}
  Deux forces agissent sur un objet:
  \begin{align*}
    \vec{F}_1 &= (3 \xhat - 7 \yhat) N \\
    \vec{F}_2 &= (- \xhat + 5 \yhat) N
  \end{align*}
  Pour une raison quelconque, la deuxième force est soudainement quadruplée et
  son orientation est inversée.  Calculer la force totale qui agit maintenant
  sur l'objet, c'est-à-dire, calculer
  \[
    \vec{F}_1 - 4\vec{F}_2
  \]
\end{frame}

\end{document}

