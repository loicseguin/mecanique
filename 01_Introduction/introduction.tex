\chapter{Introduction}

Avant de pouvoir s'attaquer à la résolution de problèmes de physique, il faut
s'armer de quelques outils.  Les deux premiers chapitres du cours visent à
introduire (ou à rappeler) les outils de base nécessaires à l'étude de la
physique.

Ces chapitres ne sont pas les plus palpitants, mais ils sont absolument vitaux
pour le reste du cours.  Comme un apprenti ébéniste qui doit apprendre à
manipuler un marteau et un rabot avant de pouvoir construire un meuble,
\marginnote{Il faut remplir son coffre à outils.}
l'étudiant en physique doit apprendre à utiliser les unités, la notation
scientifique et les vecteurs avant de pouvoir calculer l'énergie requise pour
envoyer une fusée dans l'espace.


\section{Les quatre forces fondamentales}

La physique étudie la matière et les interactions auxquelles elle est soumise.
Par exemple, pourquoi ne passez-vous pas à travers le plancher?  On sait, grâce
à des expériences réalisées au début du vingtième siècle par Lord Ernest
Rutherford \marginnote{Travail à McGill, Baron de Nelson, ``Science: physique et
philatélie''} que l'atome est principalement composé de vide : la matière
représente moins de \SI{0.001}{\percent} du volume d'un atome.  Si les atomes
sont si \emph{vides}, pourquoi ne passent-ils tout simplement pas l'un à
travers de l'autre?  Il doit y avoir une \textbf{force} qui agit pour les en
empêcher...

\emph{Toutes} les interactions observées dans l'univers sont dues à l'une ou
l'autre des quatres forces fondamentales :

\begin{itemize}
  \item force nucléaire forte
  \item force nucléaire faible
  \item force électromagnétique
  \item force gravitationnelle
\end{itemize}

Les interactions nucléaires fortes et faibles sont responsable de la cohésion
du noyau atomique et de la radioactivité.  La portée de des interactions est
inférieure à \SI{1e-17}{\meter}, soit de l'ordre de grandeur d'un noyau
atomique.

Les particules qui ont une charge électrique interagissent via la force
électromagnétique.  La grande majorité des interactions auxquelles nous sommes
habituées sont dues à la force électromagnétique.  Si vous ne passez pas à
travers du plancher, c'est parce que les électrons sous vos pieds repoussent
les électrons à la surface du plancher.  Le même phénomène explique que vous
soyez en mesure de prendre des objets et de pousser des portes.  La force
électromagnétique est également responsable des réactions chimiques, de la
lumière et de la viscosité de l'eau.

Les particules qui ont une masse interagissent via la force gravitationnelle.
Cette force, à la différence de la force électromagnétique, est toujours
attractive.  La gravité vous retient à la surface de la Terre, force la Terre à
demeurer en orbite autour du Soleil, maintient toutes les étoiles de la galaxie
ensembles et est responsable de la structure à grande échelle de l'Univers.
Une pomme tombe vers le centre de la Terre à cause de la gravité et elle
s'écrase sur le sol à cause de la force électromagnétique.

Dans le cours de mécanique, nous nous concentrerons sur le mouvement des objets
solides et sur les lois dynamique qui régissent ces mouvements.  L'état de
mouvement d'un objet change s'il est soumis à une \textbf{force} d'un des quatre
types mentionnés plus haut.  Le lois qui décrivent ces forces elles-mêmes ne
seront pas étudiées (sauf pour la gravité), mais nous développeront les outils
qui permettent de comprendre l'effet d'une force.

À plusieurs reprises, nous parlerons de \textbf{particule}.  Une particule peut
être soit une \textbf{particule élémentaire} (par exemple, un électron, un
quark, un gluon, etc.) ou un objet dont les dimensions et les mouvements
internes sont négligeables dans le contexte du problème.  Souvent, une
particule est donc un modèle abstrait d'un objet matériel qu'on utilise pour
simplifier l'analyse.  Par exemple, dans l'étude du mouvement des planètes
autour du Soleil, la taille des planètes est complètement négligeable et on
peut les considérer comme des points.


\section{Les grandeurs fondamentales de la mécanique}

Rappelez-vous que le sujet principal du cours de mécanique est l'étude du
mouvement.  Quelles sont les grandeurs nécessaires pour décrire un mouvement?
Qu'est-ce que le mouvement?  En y réfléchissant un peu, vous arrivez
probablement à une définition qui se rapproche de la suivante : un objet est en
mouvement si sa position change au fil du temps.

D'abord, la position.  Il est plus simple de définir la position d'un objet par
rapport à un

