\documentclass{tufte-handout}

\usepackage[french]{babel}
\usepackage[utf8]{inputenc}
\usepackage[T1]{fontenc}
\usepackage{amsmath, amsthm, amsfonts}
\usepackage{siunitx}
\usepackage{tikz}
\usepackage{hyperref}
%\usepackage[backend=biber, autocite=footnote]{biblatex}
\usepackage{xcolor}
\usepackage{caption}
\usepackage{booktabs}
\usepackage{mathtools}

\tikzset{>=latex}
\usetikzlibrary{calc,decorations.pathreplacing}
\sisetup{locale=FR, per-mode=symbol}

\newcommand{\abs}[1]{\left| #1 \right|}
%\renewcommand{\vec}[1]{\ensuremath{\overrightarrow{\boldsymbol{\mathrm{ #1 }}}}}
\newcommand{\rhat}{\vec{\hat{r}}}
\newcommand{\xhat}{\vec{i}}
\newcommand{\yhat}{\vec{j}}
\newcommand{\zhat}{\vec{k}}
\newcommand{\real}{\mathbb{R}}
\newcommand{\der}[2]{\frac{\mathrm{d}#1}{\mathrm{d}#2}}
\newcommand{\pder}[2]{\frac{\partial #1}{\partial #2}}
\newcommand{\dif}{\mathrm{d}}
\newcommand{\ddif}{\,\mathrm{d}}
\newcommand{\grad}{\vec{\nabla}}
\newcommand{\exemple}[1]{\begin{fullwidth}#1\end{fullwidth}}
\newcommand{\norm}[1]{\lVert #1 \rVert}
\newcommand{\vu}{\vec{u}}
\newcommand{\vv}{\vec{v}}
\newcommand{\vr}{\vec{r}}
\newcommand{\va}{\vec{a}}
\newcommand{\vF}{\vec{F}}
\newcommand{\vecxyz}[3]{#1 \xhat + #2 \yhat + #3 \zhat}
\newcommand{\vecxy}[2]{#1 \xhat + #2 \yhat}

\theoremstyle{definition}
\newtheorem*{defn}{Definition}



\title{Exercice sur le frottement}
\date{}

\begin{document}

\maketitle
\vspace{0.5cm}

On considère deux blocs de masses $m_1 = \SI{5}{\kilogram}$ et $m_2 =
\SI{8}{\kilogram}$ reliés entre eux par une corde de masse négligeable.  La
corde passe sur une poulie de masse négligeable et sans frottement.  Les blocs
reposent sur deux plans inclinés d'angles $\theta_1 = \SI{20}{\degree}$ et
$\theta_2 = \SI{35}{\degree}$.

À l'interface entre la face inférieure des blocs et le plan incliné, le
coefficient de frottement statique est $\mu_s = \num{0.2}$ et le coefficient de
frottement cinétique est $\mu_c = \num{0.1}$.

Les blocs sont initialement au repos.

\begin{marginfigure}
  \begin{tikzpicture}[scale=0.8]
    \draw[very thick] (5, 1.9) circle (0.26);
    \fill[black!30] (0, 0) -- ++(20:5.2) -- ++(82:0.2) -- ++(0.2, 0) -- ++(-82:0.2)
                    -- ++(-35:3.1) -- cycle;
    \fill (5, 1.9) circle (1pt);
    \draw[rounded corners, very thick, rotate=20] (2, 0) rectangle node[rotate=22] {$m_1$} (3.1, 0.8);
    \draw[rounded corners, very thick, shift={(6, 1.18)}, rotate=-35] (0, 0) rectangle node[rotate=-35] {$m_2$} (1.3, 0.9);
    \draw[double, shift={(0, 0.4)}] (20:3) -- ++(20:2.17) arc (110:60:0.38) -- ++(-35:1.24);
    \draw (1, 0) arc (0:20:1);
    \node at (10:1.3) {$\theta_1$};
    \draw (7.14, 0) arc (180:145:0.5);
    \node at (6.8, 0.3) {$\theta_2$};
  \end{tikzpicture}
\end{marginfigure}

Déterminer si les blocs bougent.  Si oui, déterminer l'accélération des deux
blocs.  Déterminer la tension dans la corde qui relie les deux blocs.

\section{Solution}

Pour savoir si les blocs bougent, il faut faire la somme des forces qui
agissent sur chacun des blocs et déterminer si cette somme est le vecteur nul.
Si c'est le cas, la deuxième loi de Newton garantit que l'accélération est
nulle et que les blocs ne bougent pas.

Par contre, il est impossible de faire la somme des forces sans d'abord
connaître l'orientation du frottement.  On sait que le frottement est orienté
de telle sorte qu'il s'oppose au mouvement relatif des surfaces en contact.  Si
les blocs ont tendance à glisser vers la gauche, le frottement sera vers la
droite, et vice versa.  La première étape dans la résolution de ce problème est
donc de déterminer la direction ``naturelle`` (i.e.: sans frottement) dans
laquelle les blocs sont portés à se déplacer.  Ensuite, il faudra déterminer si
la force de frottement statique est suffisante pour empêcher les blocs de
bouger.

\paragraph{Déterminer la tendance naturelle du système}

On considère le problème en ignorant le frottement.  Puisqu'il n'y a pas de
frottement et qu'on sait qu'il n'y a pas d'accélération dans la direction
perpendiculaire aux plans inclinés, on peut considérer uniquement ce qui se
passe dans la direction parallèle au plan incliné.  La figure ci-contre montre
le système d'axe utilisé pour chacun des blocs et les forces qui agissent sur
les blocs.

\begin{marginfigure}
  \begin{tikzpicture}[scale=0.8]
    \draw[very thick] (5, 1.9) circle (0.26);
    \fill[black!30] (0, 0) -- ++(20:5.2) -- ++(82:0.2) -- ++(0.2, 0) -- ++(-82:0.2)
                    -- ++(-35:3.1) -- cycle;
    \fill (5, 1.9) circle (1pt);
    \draw[rounded corners, very thick, rotate=20] (2, 0) rectangle (3.1, 0.8);
    \draw[rounded corners, very thick, shift={(6, 1.18)}, rotate=-35] (0, 0) rectangle (1.3, 0.9);
    \draw (1, 0) arc (0:20:1);
    \node at (10:1.3) {$\theta_1$};
    \draw (7.14, 0) arc (180:145:0.5);
    \node at (7.6, 0.3) {$\theta_2$};
    \draw[->] (0.2, 2) -- ++(20:1) node[right] {$x$};
    \draw[->] (0.2, 2) -- ++(110:1) node[left] {$y$};
    \draw[->] (5.5, 3) -- ++(-35:1) node[right] {$x$};
    \draw[->] (5.5, 3) -- ++(55:1) node[left] {$y$};
    \coordinate (m1) at ($(0, 0.45) + (20:2.43)$);
    \draw[->, very thick] (m1) -- ++(0, -1.7) node[below] {$\vec{F}_{g1}$};
    \draw[->, very thick] (m1) -- ++(110:1.5) node[right] {$\vec{N}_{1}$};
    \draw[->, very thick] (m1) -- ++(20:2) node[above] {$\vec{T}_{1}$};
    \coordinate (m2) at (6.84, 1.15);
    \draw[->, very thick] (m2) -- ++(0, -1.7) node[below] {$\vec{F}_{g2}$};
    \draw[->, very thick] (m2) -- ++(55:1.5) node[right] {$\vec{N}_{2}$};
    \draw[->, very thick] (m2) -- ++(145:2) node[above] {$\vec{T}_{2}$};
  \end{tikzpicture}
\end{marginfigure}

Sur le bloc 1, la deuxième loi de Newton donne, pour la composante $x$,
\begin{align}
  F_{g1x} + T_{1x} &= m_1a_{1x} \nonumber \\
  -m_1g \sin\theta_1 + T_{1} &= m_1a_{1x}.  \label{eqn:frott_ex1_m1}
\end{align}
Sur le bloc 2, on obtient
\begin{align}
  F_{g2x} + T_{2x} &= m_2a_{2x} \nonumber\\
  m_2g \sin\theta_2 - T_{2} &= m_2a_{2x}.  \label{eqn:frott_ex1_m2}
\end{align}
Puisque la corde a une longueur constante (elle ne s'étire pas et ne peux pas
se comprimer), les deux blocs ont une accélération de même module et avec le
choix d'axe qu'on a fait,
\[
  a_{1x} = a_{2x} \equiv a_x.
\]
(Le symbole $\equiv$ est utilisé pour représenter une définition.  Ici, on
l'utilise pour définir un nouveau symbole, $a_x$, qui est égal à la composante
$x$ de l'accélération des deux blocs.)

De plus, la corde a une masse négligeable donc la tension aura le même module
partout dans la corde donc
\[
  T_1 = T_2 \equiv T.
\]
Les équations \ref{eqn:frott_ex1_m1} et \ref{eqn:frott_ex1_m2} peuvent s'écrire
\begin{align}
  -m_1g \sin\theta_1 + T &= m_1a_{x}  \label{eqn:frott_ex1_m1_s} \\
  m_2g \sin\theta_2 - T &= m_2a_{x}.  \label{eqn:frott_ex1_m2_s}
\end{align}

En additionnant les équations \ref{eqn:frott_ex1_m1_s} et
\ref{eqn:frott_ex1_m2_s},
on obtient une équation dans laquelle la seule inconnue est la composante $x$
de l'accélération:
\begin{align*}
  g(m_2\sin\theta_2 - m_1 \sin\theta_1) = (m_1 + m_2) a_x \\
  a_x = \frac{g(m_2\sin\theta_2 - m_1 \sin\theta_1)}{m_1 + m_2}.
\end{align*}
La valeur qu'on obtient est $a_x = \SI{2.17}{\meter\per\second\squared}$ ce qui
signifie que les blocs ont tendance à se déplacer vers la droite (direction des
$x$ positifs pour les deux blocs).

\paragraph{Déterminer si les blocs bougent}

La tendance naturelle des blocs est de se déplacer vers la droite donc le
frottement sera orienté vers la gauche, parallèlement aux surfaces.

\begin{marginfigure}
  \begin{tikzpicture}[scale=0.8]
    \draw[very thick] (5, 1.9) circle (0.26);
    \fill[black!30] (0, 0) -- ++(20:5.2) -- ++(82:0.2) -- ++(0.2, 0) -- ++(-82:0.2)
                    -- ++(-35:3.1) -- cycle;
    \fill (5, 1.9) circle (1pt);
    \draw[rounded corners, very thick, rotate=20] (2, 0) rectangle (3.1, 0.8);
    \draw[rounded corners, very thick, shift={(6, 1.18)}, rotate=-35] (0, 0) rectangle (1.3, 0.9);
    \draw (1, 0) arc (0:20:1);
    \node at (10:1.3) {$\theta_1$};
    \draw (7.14, 0) arc (180:145:0.5);
    \node at (7.6, 0.3) {$\theta_2$};
    \draw[->] (0.2, 2) -- ++(20:1) node[right] {$x$};
    \draw[->] (0.2, 2) -- ++(110:1) node[left] {$y$};
    \draw[->] (5.5, 3) -- ++(-35:1) node[right] {$x$};
    \draw[->] (5.5, 3) -- ++(55:1) node[left] {$y$};
    \coordinate (m1) at ($(0, 0.45) + (20:2.43)$);
    \draw[->, very thick] (m1) -- ++(0, -1.7) node[below] {$\vec{F}_{g1}$};
    \draw[->, very thick] (m1) -- ++(110:1.5) node[right] {$\vec{N}_{1}$};
    \draw[->, very thick] (m1) -- ++(20:2) node[above] {$\vec{T}_{1}$};
    \draw[->, very thick] ($(m1) - (0.1, 0.4)$) -- ++(200:1) node[above] {$\vec{f}_{1}$};
    \coordinate (m2) at (6.84, 1.15);
    \draw[->, very thick] (m2) -- ++(0, -1.7) node[below] {$\vec{F}_{g2}$};
    \draw[->, very thick] (m2) -- ++(55:1.5) node[right] {$\vec{N}_{2}$};
    \draw[->, very thick] (m2) -- ++(145:2) node[above] {$\vec{T}_{2}$};
    \draw[->, very thick] ($(m2) - (0.4, 0.2)$) -- ++(145:1) node[anchor=north east, fill=white, opacity=0.7, text opacity=1, rounded corners] {$\vec{f}_{2}$};
  \end{tikzpicture}
\end{marginfigure}

Pour savoir si les blocs bougent, il faut déterminer si la force de frottement
statique maximale est suffisante pour contrer l'accélération des blocs due à la
force nette en direction des $x$ positifs.  En appliquant la deuxième loi de
Newton en direction de l'axe $y$ pour chacun des blocs, on obtient deux
équations qui permettent de trouver le module de la force normale sur chacun
des blocs:
\begin{align*}
  N_1 - m_1g \cos\theta_1 &= 0 \\
  N_2 - m_2g \cos\theta_2 &= 0.
\end{align*}
Le côté droit de l'égalité est nul parce qu'aucun des deux blocs n'a
d'accélération en direction $y$.  Les forces normales ont donc les modules
suivants
\begin{align*}
  N_1 &= m_1g \cos\theta_1 \\
  N_2 &= m_2g \cos\theta_2
\end{align*}
ce qui permet de calculer le module de la force de frottement statique maximale
pour chacun des blocs:
\begin{align*}
  f_{s, \mathrm{max} 1} &= \mu_s m_1 g \cos\theta_1 \\
  f_{s, \mathrm{max} 2} &= \mu_s m_2 g \cos\theta_2.
\end{align*}

On applique la deuxième loi de Newton en direction $x$ pour obtenir les deux
équations suivantes:
\begin{align}
  -\mu_s m_1 g\cos\theta_1 - m_1g\sin\theta_1 + T &= m_1a_x \label{eqn:frott_ex1_m1_fs} \\
  -\mu_s m_2 g\cos\theta_2 - T + m_2g\sin\theta_2 &= m_2 a_x.  \label{eqn:frott_ex1_m2_fs}
\end{align}
La somme des deux équations donne
\begin{equation}
  -\mu_s g (m_1 \cos\theta_1 + m_2\cos\theta_2) + g (m_2 \sin\theta_2 -
  m_1\sin\theta_1) = (m_1 + m_2)a_x  \label{eqn:frott_ex1_acc_fs}
\end{equation}
d'où on tire que $a_x = \SI{0.474}{\meter\per\second\squared}$.

Ce résultat signifie que même si on appliquait le frottement statique maximum,
les blocs bougeraient.  Autrement dit, les blocs se mettront nécessairement en
mouvement et on pourra calculer l'accélération en utilisant le frottement
cinétique.


\paragraph{Déterminer l'accélération des blocs}

On reprend la même analyse que dans la section précédente sauf qu'on utilise le
coefficient de frottement cinétique plutôt que le coefficient de frottement
statique.  Les équations \ref{eqn:frott_ex1_m1_fs} et \ref{eqn:frott_ex1_m2_fs}
deviennent
\begin{align}
  -\mu_c m_1 g\cos\theta_1 - m_1g\sin\theta_1 + T &= m_1a_x \label{eqn:frott_ex1_m1_fc} \\
  -\mu_c m_2 g\cos\theta_2 - T + m_2g\sin\theta_2 &= m_2 a_x.  \label{eqn:frott_ex1_m2_fc}
\end{align}
et leur somme donne
\begin{equation}
  -\mu_c g (m_1 \cos\theta_1 + m_2\cos\theta_2) + g (m_2 \sin\theta_2 -
  m_1\sin\theta_1) = (m_1 + m_2)a_x.  \label{eqn:frott_ex1_acc_fc}
\end{equation}
Par conséquent, la composante $x$ de l'accélération des blocs est 
\begin{align*}
  a_x &= \frac{-\mu_c g (m_1 \cos\theta_1 + m_2\cos\theta_2) + g (m_2 \sin\theta_2 -
  m_1\sin\theta_1)}{m_1 + m_2} \\
    &= \SI{1.32175}{\meter\per\second\squared}
\end{align*}
\[
  \boxed{a_x = \SI{1.32}{\meter\per\second\squared}}
\]


\paragraph{Déterminer le module de la tension dans la corde}

Pour déterminer le module de la tension dans la corde, il suffit de remplacer
la composante $x$ de l'accélération dans l'équation \ref{eqn:frott_ex1_m1_fc}
ou \ref{eqn:frott_ex1_m2_fc} et d'isoler $T$.
\begin{align*}
  T &= m_1a_x + \mu_c m_1 g \cos\theta_1 + m_1 g \sin\theta_1 \\
    &= \SI{27.9722}{\newton}
\end{align*}
\[
  \boxed{T = \SI{28.0}{\newton}}
\]

\end{document}
