\documentclass{tufte-handout}

\usepackage[french]{babel}
\usepackage[utf8]{inputenc}
\usepackage[T1]{fontenc}
\usepackage{amsmath, amsthm, amsfonts}
\usepackage{siunitx}
\usepackage{tikz}
\usepackage{hyperref}
%\usepackage[backend=biber, autocite=footnote]{biblatex}
\usepackage{xcolor}
\usepackage{caption}
\usepackage{booktabs}
\usepackage{mathtools}

\tikzset{>=latex}
\usetikzlibrary{calc,decorations.pathreplacing}
\sisetup{locale=FR, per-mode=symbol}

\newcommand{\abs}[1]{\left| #1 \right|}
%\renewcommand{\vec}[1]{\ensuremath{\overrightarrow{\boldsymbol{\mathrm{ #1 }}}}}
\newcommand{\rhat}{\vec{\hat{r}}}
\newcommand{\xhat}{\vec{i}}
\newcommand{\yhat}{\vec{j}}
\newcommand{\zhat}{\vec{k}}
\newcommand{\real}{\mathbb{R}}
\newcommand{\der}[2]{\frac{\mathrm{d}#1}{\mathrm{d}#2}}
\newcommand{\pder}[2]{\frac{\partial #1}{\partial #2}}
\newcommand{\dif}{\mathrm{d}}
\newcommand{\ddif}{\,\mathrm{d}}
\newcommand{\grad}{\vec{\nabla}}
\newcommand{\exemple}[1]{\begin{fullwidth}#1\end{fullwidth}}
\newcommand{\norm}[1]{\lVert #1 \rVert}
\newcommand{\vu}{\vec{u}}
\newcommand{\vv}{\vec{v}}

\theoremstyle{definition}
\newtheorem*{defn}{Definition}



\title{Devoir 2}
\date{À remettre le mercredi 26 mars 2014, au début du cours.}

\begin{document}

\maketitle
\vspace{0.5cm}

Monsieur S. glisse dans une soucoupe  sur une pente enneigée.  Il commence à
glisser d'une hauteur de \SI{15}{\meter} sur une pente faisant un angle de
\SI{35}{\degree} avec l'horizontale.  Lorsqu'il atteint le bas de la bute, il
remonte le long d'une pente faisant un angle de \SI{15}{\degree} avec
l'horizontale jusqu'à ce qu'il frappe un arbre à une hauteur de \SI{3}{\meter}.

\begin{marginfigure}
\begin{tikzpicture}[scale=1.0]
  \draw[rounded corners] (0, 3) -- (0.3, 3) -- ++(-35:3);
  \draw ($(0.3, 3) + (-35:3)$) arc (-125:-80:0.8) -- ++(15:2);
  \begin{scope}[shift={(0.7, 2.82)}, rotate=-35, scale=0.5]
    \draw (-0.5, 0) arc (-120:-60:1);
    \begin{scope}[shift={(-0.25, 0.33)}]
      \draw (0, 0.1) -- (0, -0.4) -- (0.25, -0.15) -- (0.5, -0.4);
      \draw (0, 0.2) circle (3pt);
      \draw (-0.3, 0.1) -- (0, -0.1) -- (0.3, 0.1);
    \end{scope}
  \end{scope}
  \begin{scope}[shift={(0.3, 0.06)}]
    \draw[rounded corners] (4.5, 1.5) -- (4.6, 1.6) -- (4.6, 2.4);
    \draw[rounded corners] (4.9, 1.6) -- (4.8, 1.7) -- (4.8, 2.4);
    \foreach \i in {1, ..., 100} {
      \draw ($(4.7, 2.8) + 0.5*(rand, rand)$) -- ++($0.5*(rand, rand)$);
    }
  \end{scope}
  \draw[densely dashed] (3, 1.15) -- (2, 1.15);
  \draw (2.5, 1.15) arc (180:150:0.5);
  \node at (2.3, 1.3) {\SI{35}{\degree}};
  \draw[densely dashed] (3.4, 1.15) -- (4.4, 1.15);
  \draw (4.1, 1.15) arc (0:15:0.7);
  \node at (4.5, 1.3) {\SI{15}{\degree}};
  \draw[|<->|] (0.2, 2.9) -- node[fill=white] {$h_0 = \SI{15}{\meter}$} (0.2, 1.15);
  \draw[|<->|] (5.2, 1.6) -- node[right] {$h_f = \SI{3}{\meter}$} (5.2, 1.15);
\end{tikzpicture}
\end{marginfigure}


Au bas de la pente, on peut supposer que la trajectoire de Monsieur S. est
circulaire et que seul l'orientation de sa vitesse change, pas son module.

Le coefficient de frottement cinétique pour l'interface soucoupe-pente est
$\mu_c = \num{0.3}$ et la masse de Monsieur S. est $m = \SI{62}{\kilo\gram}$.

Déterminer le module de la vitesse de Monsieur S. au moment où il entre en
collision avec l'arbre.


\end{document}

