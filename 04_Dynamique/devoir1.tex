\documentclass{tufte-handout}

\usepackage{etoolbox}
\newtoggle{answer}
\toggletrue{answer}
\usepackage[french]{babel}
\usepackage[utf8]{inputenc}
\usepackage[T1]{fontenc}
\usepackage{amsmath, amsthm, amsfonts}
\usepackage{siunitx}
\usepackage{tikz}
\usepackage{hyperref}
%\usepackage[backend=biber, autocite=footnote]{biblatex}
\usepackage{xcolor}
\usepackage{caption}
\usepackage{booktabs}
\usepackage{mathtools}

\tikzset{>=latex}
\usetikzlibrary{calc,decorations.pathreplacing}
\sisetup{locale=FR, per-mode=symbol}

\newcommand{\abs}[1]{\left| #1 \right|}
%\renewcommand{\vec}[1]{\ensuremath{\overrightarrow{\boldsymbol{\mathrm{ #1 }}}}}
\newcommand{\rhat}{\vec{\hat{r}}}
\newcommand{\xhat}{\vec{i}}
\newcommand{\yhat}{\vec{j}}
\newcommand{\zhat}{\vec{k}}
\newcommand{\real}{\mathbb{R}}
\newcommand{\der}[2]{\frac{\mathrm{d}#1}{\mathrm{d}#2}}
\newcommand{\pder}[2]{\frac{\partial #1}{\partial #2}}
\newcommand{\dif}{\mathrm{d}}
\newcommand{\ddif}{\,\mathrm{d}}
\newcommand{\grad}{\vec{\nabla}}
\newcommand{\exemple}[1]{\begin{fullwidth}#1\end{fullwidth}}
\newcommand{\norm}[1]{\lVert #1 \rVert}
\newcommand{\vu}{\vec{u}}
\newcommand{\vv}{\vec{v}}
\newcommand{\vr}{\vec{r}}
\newcommand{\va}{\vec{a}}
\newcommand{\vF}{\vec{F}}
\newcommand{\vecxyz}[3]{#1 \xhat + #2 \yhat + #3 \zhat}
\newcommand{\vecxy}[2]{#1 \xhat + #2 \yhat}

\theoremstyle{definition}
\newtheorem*{defn}{Definition}



\title{Devoir 1}
\date{À remettre le lundi 20 octobre 2014, au début du cours.}

\begin{document}

\maketitle
\vspace{0.5cm}

Monsieur S. glisse dans une soucoupe sur une pente enneigée.  Il commence à
glisser d'une hauteur de \SI{15}{\meter} sur une pente faisant un angle de
\SI{35}{\degree} avec l'horizontale.  Lorsqu'il atteint le bas de la bute, il
remonte le long d'une pente faisant un angle de \SI{15}{\degree} avec
l'horizontale jusqu'à ce qu'il frappe un arbre à une hauteur de \SI{3}{\meter}.

\begin{marginfigure}
\begin{tikzpicture}[scale=1.0]
  \draw[rounded corners] (0, 3) -- (0.3, 3) -- ++(-35:3);
  \draw ($(0.3, 3) + (-35:3)$) arc (-125:-80:0.8) -- ++(15:2);
  \begin{scope}[shift={(0.7, 2.82)}, rotate=-35, scale=0.5]
    \draw (-0.5, 0) arc (-120:-60:1);
    \begin{scope}[shift={(-0.25, 0.33)}]
      \draw (0, 0.1) -- (0, -0.4) -- (0.25, -0.15) -- (0.5, -0.4);
      \draw (0, 0.2) circle (3pt);
      \draw (-0.3, 0.1) -- (0, -0.1) -- (0.3, 0.1);
    \end{scope}
  \end{scope}
  \begin{scope}[shift={(0.3, 0.06)}]
    \draw[rounded corners] (4.5, 1.5) -- (4.6, 1.6) -- (4.6, 2.4);
    \draw[rounded corners] (4.9, 1.6) -- (4.8, 1.7) -- (4.8, 2.4);
    \foreach \i in {1, ..., 100} {
      \draw ($(4.7, 2.8) + 0.5*(rand, rand)$) -- ++($0.5*(rand, rand)$);
    }
  \end{scope}
  \draw[densely dashed] (3, 1.15) -- (2, 1.15);
  \draw (2.5, 1.15) arc (180:150:0.5);
  \node at (2.3, 1.3) {\SI{35}{\degree}};
  \draw[densely dashed] (3.4, 1.15) -- (4.4, 1.15);
  \draw (4.1, 1.15) arc (0:15:0.7);
  \node at (4.5, 1.3) {\SI{15}{\degree}};
  \draw[|<->|] (0.2, 2.9) -- node[fill=white] {$h_0 = \SI{15}{\meter}$} (0.2, 1.15);
  \draw[|<->|] (5.2, 1.6) -- node[right] {$h_f = \SI{3}{\meter}$} (5.2, 1.15);
\end{tikzpicture}
\end{marginfigure}


Au bas de la pente, on peut supposer que la trajectoire de Monsieur S. est
circulaire et que seul l'orientation de sa vitesse change, pas son module.

La masse de Monsieur S. est $m = \SI{62}{\kilo\gram}$.

Déterminer le module de la vitesse de Monsieur S. au moment où il entre en
collision avec l'arbre.


\iftoggle{answer}{

  \begin{section}{Solution}

  On divise le problème en deux parties : la descente et la montée. La vitesse
  finale pour la descente aura le même module que la vitesse initiale pour la
  montée puisqu'on dit que dans la partie circulaire en bas, seule la
  direction de la vitesse change, pas sa grandeur.

  \end{section}

  \begin{subsection}{Descente}

    La vitesse initiale au sommet de la pente est nulle. Si on peut déterminer
    l'accélération de Monsieur S., on pourra utiliser une formule de
    cinématique pour déterminer la vitesse en bas de la pente. Déterminer
    l'accélération requiert de faire l'analyse des forces qui agissent sur
    Monsieur S. et d'utiliser la deuxième loi de Newton.

    \begin{marginfigure}
    \begin{tikzpicture}[scale=1.0]
      \draw[rounded corners] (0, 3) -- (0.3, 3) -- ++(-35:3);
      \draw ($(0.3, 3) + (-35:3)$) arc (-125:-80:0.8);
      \begin{scope}[shift={(0.7, 2.82)}, rotate=-35, scale=0.5]
        \draw (-0.5, 0) arc (-120:-60:1);
        \begin{scope}[shift={(-0.25, 0.33)}]
          \draw (0, 0.1) -- (0, -0.4) -- (0.25, -0.15) -- (0.5, -0.4);
          \draw (0, 0.2) circle (3pt);
          \draw (-0.3, 0.1) -- (0, -0.1) -- (0.3, 0.1);
        \end{scope}
      \end{scope}
      \draw[densely dashed] (3, 1.15) -- (2, 1.15);
      \draw (2.5, 1.15) arc (180:150:0.5);
      \node at (2.3, 1.3) {\SI{35}{\degree}};
      \begin{scope}[shift={(2, 3)}, rotate=-35, scale=0.5]
        \draw[->] (0, 0) -- (2, 0) node[below] {$x$};
        \draw[->] (0, 0) -- (0, 2) node[left] {$y$};
      \end{scope}
      \draw[->, ultra thick] (0.7, 2.82) -- (0.7, 1) node[left] {$\vec{F}_g$};
      \draw[->, ultra thick] (0.7, 2.82) -- ++(55:1.4) node[anchor=south east] {$\vec{N}$};
    \end{tikzpicture}
    \end{marginfigure}

    Il n'y a que deux forces qui agissent sur Monsieur S. : la force exercée
    par la surface de glissade et la force gravitationnelle. En utilisant le
    système d'axes ci-contre, on peut déterminer les composantes de la force
    nette.
    \begin{eqnarray*}
      F_x &=& mg \sin \SI{35}{\degree} \\
      F_y &=& -mg \cos \SI{35}{\degree} + N
    \end{eqnarray*}
    La deuxième loi de Newton
    \begin{equation*}
      \vF = m\va
    \end{equation*}
    peut aussi s'écrire en terme des composantes ce qui donne les deux
    équations suivantes
    \begin{align}
      mg \sin \SI{35}{\degree} = ma_x      \label{eq:forcex} \\
      -mg \cos \SI{35}{\degree} + N = ma_y \label{eq:forcey}
    \end{align}
    En direction $y$, l'accélération est nécessairement nulle parce que
    Monsieur S. ne s'élève pas au-dessus de la pente et ne s'enfonce pas dans
    celle-ci. Par conséquent, le côté droit de l'équation \ref{eq:forcey} et
    égal à zéro on peut déterminer le module de la force normale à partir de
    cette équation. Dans ce problème particulier, la force normale ne nous
    intéresse pas, donc on ne fait pas le calcul.

    L'équation \ref{eq:forcex} permet de calculer la composante $x$ de
    l'accélération
    \begin{equation}
      \label{eq:accx}
      a_x = g\sin\SI{35}{\degree}.
    \end{equation}
    Il est intéressant de noter que l'accélération est indépendante de la
    masse.

    On peut maintenant utiliser l'équation de cinématique
    \begin{equation*}
      v_x^2 = v_{x0}^2 + 2a_x(x - x_0)
    \end{equation*}
    pour trouver le module de la vitesse au bas de la pente.  Pour ce calcul,
    on place l'origine du système d'axes au sommet de la pente de telle sorte
    que la position initiale soit nulle. À cette position, la vitesse est aussi
    nulle et l'équation ci-dessus devient simplement
    \begin{equation}
      \label{eq:vaax}
      v_x^2 = 2a_x x.
    \end{equation}
    \begin{marginfigure}
    \begin{tikzpicture}[scale=1.0]
      \draw[rounded corners] (0, 3) -- (0.3, 3) -- ++(-35:3);
      \draw ($(0.3, 3) + (-35:3)$) arc (-125:-80:0.8);
      \begin{scope}[shift={(0.7, 2.82)}, rotate=-35, scale=0.5]
        \draw (-0.5, 0) arc (-120:-60:1);
        \begin{scope}[shift={(-0.25, 0.33)}]
          \draw (0, 0.1) -- (0, -0.4) -- (0.25, -0.15) -- (0.5, -0.4);
          \draw (0, 0.2) circle (3pt);
          \draw (-0.3, 0.1) -- (0, -0.1) -- (0.3, 0.1);
        \end{scope}
      \end{scope}
      \draw[densely dashed] (3, 1.15) -- (2, 1.15);
      \draw (2.5, 1.15) arc (180:150:0.5);
      \node at (2.3, 1.3) {\SI{35}{\degree}};
      \draw[ultra thick] (0.7, 1.13) -- ++(2.4, 0) --
            node[anchor=south west] {$x$} (0.7, 2.82) --
            node[left] {$h_0$} cycle;
    \end{tikzpicture}
    \end{marginfigure}
    Comme la vitesse d'intérêt est celle en bas de la pente, la valeur de $x$
    doit être celle qui correspond à cette position. À partir du triangle
    rectangle dessiné ci-contre, on voit que la valeur de $x$ est donnée par
    \begin{equation}
      \label{eq:xtriang}
      x = \frac{h_0}{\sin\SI{35}{\degree}} 
    \end{equation}
    
    En combinant les équations \ref{eq:accx}, \ref{eq:vaax} et
    \ref{eq:xtriang}, on obtient
    \begin{align*}
      v_x = \pm\sqrt{\frac{2h_0g\sin\SI{35}{\degree}}{\sin\SI{35}{\degree}}} \\
      v_x = \pm \sqrt{2h_0g}
    \end{align*}
    On sait que Monsieur S. glisse vers le bas de la pente, soit vers les $x$
    positifs, donc la composante $x$ de la vitesse doit être positive. On ne
    conserve donc que la solution positive
    \begin{equation*}
      v_x = \sqrt{2h_0g}.
    \end{equation*}
    De plus, comme la vitesse est parallèle à la pente, sa composante $y$ est
    nécessairement nulle et le module de la vitesse est tout simplement la
    valeur absolue de la composante $x$
    \begin{equation}
    \label{eq:vfdescente}
      v = \sqrt{2h_0g}.
    \end{equation}
    On remarque que le module de la vitesse au bas de la pente est indépendant
    de l'angle que fait la pente avec l'horizontale.
  \end{subsection}



  \begin{subsection}{Remontée}
    La vitesse initiale pour la phase de remontée est de même module que celui
    de la vitesse finale pour la phase de descente donc
    \begin{equation*}
      v_0 = \sqrt{2h_0g}.
    \end{equation*}
    L'analyse des forces se fait de façon analogue à celle effectuée pour la
    descente.

    \begin{marginfigure}
    \begin{tikzpicture}[scale=1.0]
      \draw ($(0.3, 3) + (-35:3)$) arc (-125:-80:0.8) -- ++(15:2);
      \begin{scope}[shift={(3.4, 1.25)}, rotate=15, scale=0.5]
        \draw (-0.5, 0) arc (-120:-60:1);
        \begin{scope}[shift={(-0.25, 0.33)}]
          \draw (0, 0.1) -- (0, -0.4) -- (0.25, -0.15) -- (0.5, -0.4);
          \draw (0, 0.2) circle (3pt);
          \draw (-0.3, 0.1) -- (0, -0.1) -- (0.3, 0.1);
        \end{scope}
      \end{scope}
      \begin{scope}[shift={(5, 2)}, rotate=15, scale=0.5]
        \draw[->] (0, 0) -- (2, 0) node[below] {$x$};
        \draw[->] (0, 0) -- (0, 2) node[left] {$y$};
      \end{scope}
      \draw[densely dashed] (3.4, 1.15) -- (4.4, 1.15);
      \draw (4.1, 1.15) arc (0:15:0.7);
      \node at (4.5, 1.3) {\SI{15}{\degree}};
      \draw[->, ultra thick] (3.4, 1.25) -- ++(0, -1.5) node[left] {$\vec{F}_g$};
      \draw[->, ultra thick] (3.4, 1.25) -- ++(105:1.4) node[anchor=south east] {$\vec{N}$};
    \end{tikzpicture}
    \end{marginfigure}

    On ne s'intéresse encore une fois qu'à la composante $x$ de la force nette
    \begin{eqnarray*}
      F_x &=& -mg \sin \SI{15}{\degree} \\
    \end{eqnarray*}
    La deuxième loi de Newton donne
    \begin{align*}
      -mg \sin \SI{15}{\degree} = ma_x
    \end{align*}

    La composante $x$ de l'accélération est donc
    \begin{equation}
      \label{eq:accxmonte}
      a_x = -g\sin\SI{15}{\degree}.
    \end{equation}
    Il est intéressant de noter que l'accélération est indépendante de la
    masse.

    On peut maintenant utiliser l'équation de cinématique
    \begin{equation*}
      v_x^2 = v_{x0}^2 + 2a_x(x - x_0)
    \end{equation*}
    pour trouver le module de la vitesse lorsque Monsieur S. heurte l'arbre.
    Pour ce calcul, on place l'origine du système d'axes au bas de la pente de
    telle sorte que la position initiale soit nulle. À cette position, la
    vitesse a le même module que la vitesse finale de la descente et l'équation
    ci-dessus devient
    \begin{equation}
      \label{eq:vaaxmonte}
      v_x^2 = 2h_0g + 2a_x x.
    \end{equation}
    \begin{marginfigure}
    \begin{tikzpicture}[scale=1.0]
      \draw ($(0.3, 3) + (-35:3)$) arc (-125:-80:0.8) -- ++(15:2);
      \begin{scope}[shift={(3.4, 1.25)}, rotate=15, scale=0.5]
        \draw (-0.5, 0) arc (-120:-60:1);
        \begin{scope}[shift={(-0.25, 0.33)}]
          \draw (0, 0.1) -- (0, -0.4) -- (0.25, -0.15) -- (0.5, -0.4);
          \draw (0, 0.2) circle (3pt);
          \draw (-0.3, 0.1) -- (0, -0.1) -- (0.3, 0.1);
        \end{scope}
      \end{scope}
      \draw[densely dashed] (3.4, 1.15) -- (4.4, 1.15);
      \draw (4.1, 1.15) arc (0:15:0.7);
      \node at (4.5, 1.3) {\SI{15}{\degree}};
      \draw[ultra thick] (3.4, 1.15) -- ++(2, 0) --
              node[right] {$h_f$} ++(0, 0.56) --
              node[anchor=south east] {$x$} cycle;
    \end{tikzpicture}
    \end{marginfigure}
    À partir du triangle
    rectangle dessiné ci-contre, on voit que la valeur de $x$ à la position de
    l'arbre est donnée par
    \begin{equation}
      \label{eq:xtriangmonte}
      x = \frac{h_f}{\sin\SI{15}{\degree}} 
    \end{equation}
    
    En combinant les équations \ref{eq:accxmonte}, \ref{eq:vaaxmonte} et
    \ref{eq:xtriangmonte}, on obtient
    \begin{align*}
      v_x^2 &= 2h_0g - \frac{2h_fg\sin\SI{15}{\degree}}{\sin\SI{15}{\degree}} \\
      v_x^2 &= 2g(h_0 - h_f) \\
      v_x &= \pm\sqrt{2g(h_0 - h_f)}
    \end{align*}
    On sait que Monsieur S. glisse vers le haut de la pente, soit vers les $x$
    positifs, donc la composante $x$ de la vitesse doit être positive. On ne
    conserve donc que la solution positive
    \begin{equation*}
      v_x = \sqrt{2g(h_0 - h_f)}.
    \end{equation*}
    De plus, comme la vitesse est parallèle à la pente, sa composante $y$ est
    nécessairement nulle et le module de la vitesse est tout simplement la
    valeur absolue de la composante $x$
    \begin{equation}
    \label{eq:vfdescente}
      v = \sqrt{2g(h_0 - h_f)}.
    \end{equation}
    Ce résultat ne dépend que de la hauteur initiale, de la hauteur finale et
    de l'accélération gravitationnelle terrestre!

    Le calcul de la valeur numérique donne
    \begin{align*}
      v = \sqrt{2 \times \SI{9.8}{\meter\per\second\squared} \times
          (\SI{15}{m} - \SI{3}{m})}
    \end{align*}
    \begin{equation*}
      \boxed{v = \SI{15.3}{\meter\per\second}}
    \end{equation*}
    Ce module de vitesse est très élevé. Une collision entre un être humain
    sans protection et un arbre à cette vitesse serait probablement fatale.
    Cette valeur élevée est probablement due au fait que le frottement entre la
    soucoupe et la surface de glissade est complètement négligée dans ce
    problème.
  \end{subsection}

  \begin{subsection}{Remarques}
    La résolution de ce problème montre que dans une situation qui n'implique
    que la force de gravité, le module de la vitesse finale est complètement
    indépendant de la trajectoire exacte suivie par un corps. Il est déterminé
    uniquement par la hauteur initiale et la hauteur finale du corps. Ce
    résultat réapparaîtra lorsqu'on étudiera les concepts de travail et
    d'énergie.
  \end{subsection}
}

\end{document}

