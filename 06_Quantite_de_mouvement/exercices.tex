\documentclass{beamer}


\usepackage[french]{babel}
\usepackage[utf8]{inputenc}
\usepackage[T1]{fontenc}
\usepackage{amsmath, amsthm, amsfonts}
\usepackage{siunitx}
\usepackage{tikz}
\usepackage{hyperref}
%\usepackage[backend=biber, autocite=footnote]{biblatex}
\usepackage{xcolor}
\usepackage{caption}
\usepackage{booktabs}
\usepackage{mathtools}

\tikzset{>=latex}
\usetikzlibrary{calc,decorations.pathreplacing}
\sisetup{locale=FR, per-mode=symbol}

\newcommand{\abs}[1]{\left| #1 \right|}
%\renewcommand{\vec}[1]{\ensuremath{\overrightarrow{\boldsymbol{\mathrm{ #1 }}}}}
\newcommand{\rhat}{\vec{\hat{r}}}
\newcommand{\xhat}{\vec{i}}
\newcommand{\yhat}{\vec{j}}
\newcommand{\zhat}{\vec{k}}
\newcommand{\real}{\mathbb{R}}
\newcommand{\der}[2]{\frac{\mathrm{d}#1}{\mathrm{d}#2}}
\newcommand{\pder}[2]{\frac{\partial #1}{\partial #2}}
\newcommand{\dif}{\mathrm{d}}
\newcommand{\ddif}{\,\mathrm{d}}
\newcommand{\grad}{\vec{\nabla}}
\newcommand{\exemple}[1]{\begin{fullwidth}#1\end{fullwidth}}
\newcommand{\norm}[1]{\lVert #1 \rVert}
\newcommand{\vu}{\vec{u}}
\newcommand{\vv}{\vec{v}}

\theoremstyle{definition}
\newtheorem*{defn}{Definition}




\begin{document}


\begin{frame}
  \frametitle{Recul d'une arme à feu}

  Les munitions .577 Tyrannosaur sont destinées à la chasse aux éléphants et
  aux rhinocéros.  La balle qui est projetée a une masse de \SI{48.6}{\gram} et
  peut atteindre une vitesse de \SI{750}{\meter\per\second}.

  Le recul d'une carabine qui tire une telle munition est
  \textbf{\href{https://www.youtube.com/watch?v=-EVqT3XEzss}{considérable}}.

  Si on tire une .577 Tyrannosaur avec un fusil A-SQUARE Hannibal de
  \SI{6.2}{\kilogram}, déterminer la vitesse de la carabine après que la balle
  soit partie.

  \uncover<2>{\alert{$v_F = \SI{5.88}{\meter\per\second}$}}

\end{frame}


\begin{frame}
  \frametitle{Pendule ballistique}

  La balle (de masse \SI{48.6}{\gram}) parcourt une certaine distance dans les
  airs puis s'enfonce dans un bloc de masse \SI{8}{\kilogram} suspendu au bout
  de deux cordes de \SI{1.5}{\meter} de longueur.  La balle s'immobilise dans
  le bloc qui monte d'une hauteur maximale de \SI{3}{\centi\meter} avant de
  redescendre vers sa position d'équilibre.

  Déterminer la vitesse de la balle juste avant sa collision avec le bloc.

  \begin{center}
    \begin{tikzpicture}[>=latex, scale=0.5]
      \draw[ultra thick] (-1, 5) -- (1, 5);
      \draw (-0.5, 5) -- (-0.5, 1);
      \draw (0.5, 5) -- (0.5, 1);
      \draw (-0.8, -1) rectangle (0.8, 1);
      \draw[densely dashed] (-0.5, 5) -- ++(-65:4);
      \draw[densely dashed] (0.5, 5) -- ++(-65:4);
      \draw[densely dashed, shift={(1.7, 0.3)}] (-0.8, -1) rectangle (0.8, 1);
      \draw[->|] (3.5, 0) -- (3.5, -0.7);
      \draw[->|] (3.5, -1.7) -- (3.5, -1);
      \node at (4, -0.85) {$h$};
    \end{tikzpicture}
  \end{center}

  \uncover<2>{\alert{$v_B = \SI{127}{\meter\per\second}$}}
\end{frame}


\begin{frame}
  \frametitle{Explosion}
\end{frame}
  

\end{document}
